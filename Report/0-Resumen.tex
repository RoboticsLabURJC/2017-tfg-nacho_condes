\chapter*{Resumen}
El \emph{reconocimiento de objetos} en imágenes ha sido un problema recurrente en la historia de la \emph{visión artificial}. Gracias a la inclusión de algoritmos basados en aprendizaje máquina y, más recientemente, de las técnicas de \emph{aprendizaje profundo}, se han enfrentado con éxito problemas como el reconocimiento de señales de tráfico o la videovigilancia. En concreto, las \emph{redes neuronales convolucionales} se han convertido en la punta de lanza de este tipo de algoritmos en los últimos años. Su eficacia en la resolución de problemas como los anteriormente mencionados resulta indiscutible en muchos casos, lo que poco a poco está favoreciendo su uso en aplicaciones comerciales. A pesar de ello, siguen siendo acusadas de actuar como una caja negra o \emph{black box}, ya que por lo general su aprendizaje es un proceso opaco y difícil de interpretar.

Por todo ello, este trabajo de fin grado tiene como metas el \emph{estudio detallado} de las redes neuronales convolucionales y su \emph{aplicación} en el abordaje de un determinado problema. En este sentido, se desarrollará un \emph{clasificador de dígitos manuscritos en tiempo real}. Para entrenar e implementar las redes neuronales convolucionales, se empleará la plataforma \emph{Keras}. El proyecto comienza con el análisis de una red neuronal convolucional de ejemplo proporcionada por dicha plataforma. Posteriormente, se procede al desarrollo del \emph{componente} clasificador de dígitos. Este componente adquiere imágenes desde una fuente de vídeo, las clasifica gracias a una red neuronal de Keras y muestra el resultado en una interfaz gráfica. Además, se ha conformado un \emph{banco de pruebas} en el que se incluyen bases de datos para alimentar las redes neuronales convolucionales y herramientas para calcular y visualizar parámetros de evaluación. Por último, gracias a las herramientas del banco de pruebas, se discutirán los \emph{efectos que produce el aprendizaje sobre el desempeño} de distintas redes neuronales, y aquella que mejores resultados arroje será integrada en el componente clasificador de dígitos para lograr una mayor robustez.

Los resultados obtenidos ponen de manifiesto el gran potencial de las redes neuronales convoluciones y proyectan algo de luz sobre su aprendizaje, dejando abierta la puerta a su empleo en la resolución de problemas más complejos.