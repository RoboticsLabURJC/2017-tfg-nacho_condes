\documentclass[11pt, a4paper]{svproc}
\usepackage[utf8]{inputenc}
\usepackage{graphicx}
\usepackage{multicol}
\usepackage{footmisc}

\title{Person following robot behavior using Deep Learning}
\author{Ignacio Condés
		\and José María Cañas}
\institute{Universidad Rey Juan Carlos}

\begin{document}
	\maketitle
	
	\begin{abstract}
		The proposed following behavior has been developed combining a \emph{perception module} and a \emph{controller module}. The perception module addresses person detection using a \emph{Convolutional Neural Network} to process the incoming images. Its main benefit over traditional methods is the \emph{robustness}, but care has to be put on not losing a real-time operation. This module uses a pretrained TensorFlow SSD person detection CNN. In order to increase the robustness, we have also implemented a \emph{person tracker} that eliminates the effect of false positives/negatives. It also has a \emph{face detection} block. The extracted faces are analyzed by a \emph{siamese network}, in order to reidentify the tracked individual person. This perception module tells which person in the image is the one to follow, even on poorly lightened scenarios. This is combined with depth readings, to obtain the relative position of the person. The control module implements a case-based \emph{PID} controller for a reactive smooth response, moving the robot towards the target person. The entire system has been experimentally validated on a real Turtlebot robot, with an Asus Xtion conventional RGBD camera.
	\end{abstract}

	
	
\end{document}