\documentclass[11pt, a4paper]{svproc}
\usepackage[utf8]{inputenc}
\usepackage{graphicx}
\usepackage{multicol}
\usepackage{footmisc}
\usepackage{url}
\def\UrlFont{\rmfamily}

\title{Person following robot behavior using Deep Learning}
\author{Ignacio Condés\inst{1} \and José María Cañas\inst{2}}

\institute{Universidad Rey Juan Carlos,
	\email{ignacio.condes.m@gmail.com},
	\and
Universidad Rey Juan Carlos, \email{jmplaza@gsyc.es}}




\begin{document}
	\maketitle
	
	\begin{abstract}
		The proposed following behavior has been developed combining a \emph{perception} and \emph{control} modules. The perception module addresses person detection on the incoming images using a pretrained TensorFlow SSD person detection \emph{Convolutional Neural Network}, offering \emph{robustness} over traditional methods, although care has to be put on not losing real-time operation. This is also supported by a \emph{person tracker} that eliminates the effect of false positives/negatives. It also has a \emph{face detection} block, which extracts faces to be analyzed by a \emph{siamese network}, in order to reidentify the tracked individual person. This perception module tells which person in the image is the one to follow, even on poorly lightened scenarios. This is combined with depth readings, to obtain the relative position of the person. The control module implements a case-based \emph{PID} controller for a reactive smooth response, moving the robot towards the target person. The entire system has been experimentally validated\footnote{\url{https://jderobot.org/Naxvm-tfg}} on a real Turtlebot robot, with an Asus Xtion RGBD camera.
	\end{abstract}

	
	
\end{document}