\section{Infrastructure}

On this article, we rely on a certain set of hardware/software elements:

\subsection*{Hardware}

\begin{itemize}
	\item \emph{Asus Xtion Pro Live}: RGBD (RGB + Depth) sensor. It counts with a regular digital camera, and with an infrared emitter, which radiates IR beams. Their reflections are retrieved by a depth sensor, which generates a \emph{point cloud} with the measured distance to each reflection surface. A \emph{registration} process is performed in the driver, to project these points into the RGB pixels (due to the spacial offset between both sensors).\\
	
	\item \emph{Turtlebot 2}: 2-wheeled robot, designed for didactic purposes. It counts with 2 freedom degrees on its motors: \emph{angular} movements and \emph{linear} movements.\\
\end{itemize}

\subsection*{Software}

\begin{itemize}
	\item \emph{ROS}: robotics software development framework. It provides, among others, the low level drivers to get correctly communicated with the hardware devices.\\
	
	\item \emph{JdeRobot}: research and educational robotics framework. The taken benefit from this framework is the higher level interfaces to communicate our developed program with the drivers created by ROS.\\
	
	
	\item \emph{TensorFlow}: \emph{deep learning} library recently developed by Google, and already widely adopted by big companies. It operates with \emph{tensors}\footnote{Generic data structure with a specific number of dimensions.} on an optimized low-level way. It allows GPU parallelization of the operations, which leads to a very efficient implementation of \emph{deep learning} algorithms.\\
	
	
	\item \emph{Python}: high-level interpreted programming language. It is widely used on \emph{machine learning} purposes due to its focus on easiness and \emph{Object Orientation}. It counts with implementations of the previously described libraries and frameworks.\\
\end{itemize}