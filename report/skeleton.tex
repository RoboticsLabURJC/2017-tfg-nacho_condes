\subsection{Reactive behavioral}

The next milestone was to be able to connect this intelligent perception to a reactive behavioral, translated into a robot following a certain person. This requires of two main abilities:
\begin{itemize}
	\item Object detection and discrimination: this is just what we achieved with the previous component. We only have to constrain the detections to only keep persons.
	\item People reidentification: as it will be seen, the component is able to track a certain person (which is called \emph{mom}). This is achieved with a person-by-person face comparison with a reference face (which belongs to \emph{mom}). This process is performed by a \emph{siamese neural network}, which is inspired by [tfm marcos pieras].
\end{itemize}

With these capabilities, we can perform a neural control over physical actuators. Its main avail is the significant strength that a \emph{convolutional neural network} can keep on variable light conditions, which is the main Damocles sword on the image processing field.


\subsubsection{SSD Detector}
\subsubsection{Network Architecture}
\subsubsection{Load of different models}


\chapter{\texttt{ObjectDetector} node}
\section{Description}
\section{Functional architecture}
\section{Neural Network processing}


\chapter{\texttt{FollowPerson} node}
\label{chap:followperson}
\section{Description}
\section{Functional architecture}
\section{Neural Network processing}
\section{Face detection and identification}
\section{Tracking algorithm}
\section{Physical response (\emph{PID} controller)}



\chapter{Conclusions}
	\section{Conclusions}
	\section{Future research lines}