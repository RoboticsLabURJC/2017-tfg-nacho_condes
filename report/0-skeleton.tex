\chapter{Introduction}
Here, we will talk about current robotics applications, which can be really useful at daily tasks. These tasks are of greater interest when the behavioral of a robot tends to emulate the human one (without the human body limitations). This requires a polished (and somehow complex) behavioral.\\

But, almost every time, a behavioral is a reactive response, triggered by a certain input (typically perceived by on-board sensors, among others). This raw data is generally mapped into a concrete response. At this point, we can take much benefit from \emph{Neural Networks}, which can yield a really effective, fast and \textbf{strong} response, specially when we are dealing with image processing.\\

There are several processing tools based on Neural Networks at the JdeRobot robotics platform (Detection Suite, TFG David, Nuria, PFC Roberto Calvo [this one does not implement NNs]), which as we have just mentioned, can make such a brilliant tandem along with a reactive behavioral.\\

As we will be able to see later, \textbf{robotics + deep learning rocks!}

\chapter{Infrastructure}

\section{TensorFlow}
\subsection{CUDA}
\section{Keras}
\section{JdeRobot}
\subsection{Digit Classifier}
\subsection{Detection Suite}
\section{OpenCV}
\section{ROS}
\subsection{OpenNIServer}
\section{Hardware}
\subsection{Asus Xtion}
\subsection{Turtlebot / Kobuki}
\subsection{Sony EVI D100P}




\chapter{Objectives}


With all this in mind, the objective of the present work has been to get a little further on \textbf{deep learning applied to robotics}, developing two sucessive components.
\section{\texttt{ObjectDetector}}
As it will be described on the suitable chapter, we have built a component (\texttt{Object Detector}) which implements \textbf{a generic object detection structure} on an image. This component is ready to work in \emph{real time} (processing a streaming video flow).\\

\section{\texttt{FollowPerson}}

The next milestone was to be able to connect this intelligent perception to a reactive behavioral, translated into a robot following a certain person. This requires of two main abilities:
\begin{itemize}
	\item Object detection and discrimination: this is just what we achieved with the previous component. We only have to constrain the detections to only keep persons.
	\item People reidentification: as it will be seen, the component is able to track a certain person (which is called \emph{mom}). This is achieved with a person-by-person face comparison with a reference face (which belongs to \emph{mom}). This process is performed by a \emph{siamese neural network}, which is inspired by [tfm marcos pieras].
\end{itemize}

With these capabilities, we can perform a neural control over physical actuators. Its main avail is the significant strength that a \emph{convolutional neural network} can keep on variable light conditions, which is the main Damocles sword on the image processing field.



\chapter{\texttt{Object Detector}}
\section{Description}
\section{Functional architecture}
\section{Neural Network processing}


\chapter{\texttt{Follow Person}}
\section{Description}
\section{Functional architecture}
\section{Neural Network processing}
\section{Face detection and identification}
\section{Tracking algorithm (physical response)}



\chapter{Conclusions}

\chapter{Future research lines}

\chapter{Bibliography}