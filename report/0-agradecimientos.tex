\newgeometry{textwidth=16cm,textheight=28cm,voffset=-2cm,bottom=0cm}
\chapter*{Agradecimientos}

Hace ya unas semanas, e incluso meses, que pensaba en el momento que justo ahora estoy viviendo, sentado frente a las teclas, con más café que sangre en las venas, y con un duelo interno al saber que no puedo extenderme en demasía en este capítulo.

Los últimos tiempos en mi vida han significado una etapa de cambio, de algunas decepciones pero, por fortuna, muchas más alegrías. Es increíble comprobar cómo ciertos factores consiguen moverte a luchar contra viento y marea con lo que sea, incluso ponerte a buscar, a las 4 de la mañana, la línea 524 del instanciador de TensorFlow, a ver por qué vuelve a saltarte aquella excepción de la que pensabas haberte librado hace dos días, cuando desapareció con tanta magia como cuando vino. Ha sido un trabajo arduo y sufrido, pero a la vez muy gratificante. Y esto se lo debo a mucha gente. Ojalá tuviera folios, memoria y capacidad para mentaros a todos, pero por desgracia no va a ser así, por lo que todos podéis y debéis sentiros incluidos en ese grupo de gente.\\


Para empezar, nada de esta investigación habría sido posible sin mi tutor, José María, al que le debo agradecer la fuente infinita de paciencia y comprensión que ha sido cuando las circunstancias así lo han requerido. De manera inherente a su condición de profesor, le debo una parte importante del conocimiento que he adquirido durante el transcurso de este año. No obstante, el conocimiento se puede obtener en libros, \emph{papers}, artículos... Lo que ningún recurso puede aportar es el impulso motivacional que me ha sabido inculcar. Muchas veces, he de admitir que he entrado a su despacho para reunirnos, con la mera intención de avanzar y quitarme una carga de encima entregando el trabajo acordado. No obstante, la gran mayoría de ellas he salido por la misma puerta minutos después con el brillo en los ojos de quien tiene toda la ilusión para seguir adelante, al haber visto conducidos todos mis esfuerzos hacia el máximo gradiente (valga la expresión), en el cual darlo todo.\\

Pero no todo es trabajar sin descanso, ya que también he de agradecer a la gente que ha sabido despejar pájaros en mi cabeza de toda índole (incluyendo la excepción mencionada arriba), ayudándome a respirar cuando realmente lo necesitaba para seguir adelante. Pablete, Jose, David, Javi, mi familia, y todo el mundo que ha conseguido sacarme una sonrisa, por muy breve que fuera. Quiero hacer una mención especial para mis compañeros de banda, con los cuales pasear por los pentagramas ha sido siempre una diversión infinita, que ojalá nunca acabe. Sebas, Chema, Urbano, también va por vosotros, sois los mejores.\\

En último lugar, quiero mandar mi más sincero agradecimiento a todas las personas que hacen sentir a su pareja como me hace sentir a mí la mía. Ojalá seáis mucha gente así, por fortuna la que conozco, me la he quedado yo. Me deshago en admiración y agradecimientos para ti, Almudena. Gracias por haber compartido alegrías a mi lado, y haber sufrido mis penas más cerca todavía. Ojalá pudiera cuantificar la enorme parte de este trabajo que te pertenece, el gran mérito que tienes por haberme empujado hacia delante contra viento y marea. Por supuesto también, mil gracias a tu familia, mi segunda familia, que siempre me ha insuflado fuerzas y ánimos cuando más lo necesitaba. Ojalá todos los retos de mi vida se presenten en estas condiciones, para poder abordarlos con esta fuerza y este apoyo incondicional.


\flushright Un millón de gracias a todos, sentíos parte de lo orgulloso\\ que puedo estar de este trabajo, ya que os pertenece también. \textbf{GRACIAS.}

\flushleft
\restoregeometry


